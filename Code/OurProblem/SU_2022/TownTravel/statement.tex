\documentclass[11pt,a4paper]{article}

\usepackage{../.resources/style}
 \graphicspath{ {../images/} }

\begin{document}

\begin{problem}{เดินทางข้ามชุมชน}{standard input}{standard output}{1 second}{32 megabytes}

% โจทย์ดูสร้างสรรค์ดีครับ และน่าจะเหมาะกับการสร้าง test case ง่าย ๆ สำหรับคนที่คิดวิธีที่มัน Optimal ไม่ออก ให้ได้คะแนนไปบางส่วนด้วย  แบบนี้น่าจะทำให้คะแนนกระจายได้ดีครับ

% ผมคิดว่าตัวค่า k_i มันจะเป็น trick ที่หลายคนดูไม่ออกว่าจะสามารถเอามาใช้ซ้ำได้โดยไม่ต้องสร้าง MST ใหม่ในบางกรณี ซึ่งถ้าเราเตรียมเคสที่คนที่ดูเรื่องนี้ไม่ออกไว้ด้วยก็จะทำให้คะแนนกระจายได้หลากหลาย = subtask O(NQ)

\parindent 5ex
กาลครั้งหนึ่งนานมาแล้ว มีประเทศที่ประกอบด้วยชุมชน $N$ แห่ง เชื่อมต่อกันด้วยถนนสองทิศทาง $M$ เส้น แต่ละเส้นมีความยาว $w_i$ กิโลเมตร โดยชุมชนสองแห่งใด ๆ จะสามารถเดินทางหากันได้ และจะไม่มีชุมชนสองชุมชนใด ๆ มีถนนเชื่อมติดกันเกิน 1 เส้น นอกจากนี้ในชุมชนแต่ละแห่งจะมีปั๊มน้ำมันประจำชุมชนอยู่ ที่จะสามารถเติมน้ำมันให้กับรถที่ขับผ่านมาได้เต็มถัง

คุณเป็นหัวหน้าหน่วยจู่โจมทางอากาศของรัฐบาลที่จำเป็นต้องส่งสายสืบออกเดินทางทั้งหมด $Q$ ครั้ง เพื่อตามหาชุมชนของผู้ก่อการร้ายที่หลบซ่อนอยู่ในประเทศแห่งนี้ ซึ่งสำหรับการเดินทางทุกครั้งจะมีเงื่อนไขอยู่ว่า รถที่สายสืบของคุณขับอยู่\textbf{ห้ามน้ำมันหมดกลางถนนเด็ดขาด} (น้ำมันหมดที่ชุมชนได้) และสายสืบของคุณสามารถเติมน้ำมันรถให้เต็มถังได้ทุกครั้งเมื่อเดินทางมาถึงชุมชนใด ๆ 

ด้วยความที่ในการเดินทางแต่ละครั้งคุณจะเช่ารถที่แตกต่างกันให้กับสายสืบ ความจุถังน้ำมันของรถที่เช่าก็จะแตกต่างกันไปด้วย (แต่คุณจะไม่เปลี่ยนรถที่เช่าระหว่างการเดินทาง) ดังนั้นแล้ว คุณจึงอยากทราบว่าในการเดินทางแต่ละครั้ง สายสืบของคุณจะสามารถ\textbf{เดินทางจากชุมชนหมายเลข $a_i$ ไปยังชุมชนหมายเลข $b_i$ ได้หรือไม่ หากรถที่คุณเช่ามีความจุถังน้ำมัน $k_i$ ลิตร} และการเดินทางบนถนนเส้นใด ๆ จะใช้น้ำมัน 1 ลิตร/กิโลเมตร
\begin{figure}[ht]
\centering
\includegraphics[width=0.85\textwidth]{SUCTQ1.JPG}
\end{figure}

ตัวอย่างเช่นในภาพ ประเทศของคุณมีชุมชน $5$ แห่ง หากคุณต้องการส่งสายสืบให้เดินทางจากชุมชน $0 \rightarrow 4$ ด้วยรถที่มีความจุน้ำมัน $3$ ลิตร จะสังเกตว่าสายสืบของคุณจะไม่สามารถเดินทางไปถึงได้ เพราะจะน้ำมันหมดกลางที่ถนนเส้นสีแดงก่อน แต่ถ้ารถมีความจุน้ำมัน $4$ ลิตร จะสามารถเดินทางไปถึงได้ด้วยเส้นทางสีเขียว

\InputFile
\textbf{บรรทัดแรก} จำนวนเต็ม $N, M, Q$ แทนจำนวนชุมชน จำนวนถนน และจำนวนการเดินทาง $(1\leq N \leq 100\,000, 1 \leq M \leq 200\,000, 1 \leq Q \leq 300\,000)$\\
\textbf{$M$ บรรทัดต่อมา} จำนวนเต็ม $u_i, v_i, w_i$ แทนว่ามีถนนความยาว $w_i$ กิโลเมตร เชื่อมต่อกันระหว่างชุมชนหมายเลข $u_i$ และ $v_i$ $(0 \leq u_i, v_i < N, 1 \leq w_i \leq 10^9, u \neq v)$\\
\textbf{$Q$ บรรทัดต่อมา} จำนวนเต็ม $a_i, b_i, k_i$ แทนว่าคุณต้องการส่งสายสืบเดินทางจากชุมชนหมายเลข $a_i$ ไปยังชุมชนหมายเลข $b_i$ โดยรถที่คุณเช่าจะมีความจุถังน้ำมัน $k_i$ ลิตร $(0 \leq a_i, b_i < N, 1 \leq k_i \leq 10^9, a \neq b)$

\OutputFile
ส่งออก $Q$ บรรทัด แทนคำตอบของแต่ละการเดินทาง โดยส่งออก \texttt{'Yes'} เมื่อรถของสายสืบสามารถเดินทางได้ในครั้งนั้น และส่งออก \texttt{'No'} เมื่อรถของสายสืบไม่สามารถเดินทางด้วยเงื่อนไขดังกล่าวได้

\Scoring
\begin{itemize}

\item \textbf{ปัญหาย่อยที่ 1 (25 คะแนน)} $N, M \leq 1\,000, Q \leq 5$
\item \textbf{ปัญหาย่อยที่ 2 (25 คะแนน)} $N, M \leq 1\,000$
\item \textbf{ปัญหาย่อยที่ 3 (20 คะแนน)} $1 \leq w_i, k_i \leq 2$
\item \textbf{ปัญหาย่อยที่ 4 (30 คะแนน)} ไม่มีเงื่อนไขเพิ่มเติม
\end{itemize}

\Examples

\begin{example}
\exmp{
5 5 2
0 1 4
1 3 2
1 2 3
1 4 5
3 4 1
0 4 3
0 4 4
}{
No
Yes
}\exmp{
7 8 3
0 1 4
2 5 6
4 2 4
3 1 7
4 1 3
2 1 11
3 5 10
6 5 4
5 3 8
1 6 5
0 6 7
}{
Yes
No
Yes
}%
\end{example}

\Examples

\begin{example}
\exmp{
4 5 3
0 1 1
1 2 2
2 3 1
1 4 1
0 4 2
1 4 1
0 2 2
3 1 1
}{
Yes
Yes
No
}%
\end{example}

\end{problem}

\end{document}
